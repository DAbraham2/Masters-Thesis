\chapter{Wireless protocols}
\label{chap:wireless}

bla bla

In this chapter, I will describe the Industrial, Scientific and Medical (ISM) 2,4 \(GHz\) radio band in \autoref{sec:ism}.
After that, I will demonstrate each wireless protocol starting with Bluetooth Low Energy in \autoref{sec:ble},
Zigbee in \autoref{sec:zig}, and finally Thread/OpenThread in \autoref{sec:ot}.
\autoref{sec:15_4} will provide an overview of the IEEE 802.15.4 radio specification, of which both Zigbee and Thread is based on.

\section{ISM radio band}
\label{sec:ism}

\section{Bluetooth Low Energy}
\label{sec:ble}
Bluetooth is one of the most popular commodity radios for wireless devices.
As a representative of the frequency hopping spread spectrum radios,
it is a natural alternative to broadcast radios in the context of sensor networks. \cite{Leopold03}

Bluetooth is managed by the Bluetooth Special Intrest Group (SIG),
which has more than 38,000 member companies as of 2023. \cite{bt_history}
The first specification was released in 1999.
Since then, the SIG released 5 core specifications, 5.4 being the latest adoption in 2023. \cite{bt_spec_history}

blab bla.\cite{ble_primer23}.

\subsection{Introcution}
\label{ble:int}
Bluetooth Low Energy (BLE) started as part of the Bluetooth 4.0 Core Specification.
It is tempting to represent itself as a smaller and highly optimized version of its bigger brother, classic Bluetooth.
In reality, BLE has an entirely different lineage and design goals.
Originally, it was designed by Nokia as Wibree before being adopted by the Bluetooth Special Intrest Group (SIG).
The focus was to design a radio standard with the lowest possible power consumption, specifically optimized for low cost, low bandwidth, low power, and low complexity. \cite{Townsend14}

\subsection{Overview}
\label{ble:ow}

\subsection{Generic Access Profile}
\label{ble:gap}

\subsection{Generic Attibute Profile}
\label{ble:gatt}

\section{IEEE 802.15.4}
\label{sec:15_4}

bla bla

\section{Zigbee}
\label{sec:zig}
zigbee is good.

\section{Thread/OpenThread}
\label{sec:ot}
openthread.
