\chapter{Wireless protocols}
\label{chap:wireless}

bla bla

In this chapter, I will describe the Industrial, Scientific and Medical (ISM) 2,4 \(GHz\) radio band in \autoref{sec:ism}.
After that, I will demonstrate each wireless protocol starting with Bluetooth Low Energy in \autoref{sec:ble},
Zigbee in \autoref{sec:zig}, and finally Thread/OpenThread in \autoref{sec:ot}.
\autoref{sec:15_4} will provide an overview of the IEEE 802.15.4 radio specification, of which both Zigbee and Thread is based on.

\section{ISM radio band}
\label{sec:ism}

\section{Bluetooth Low Energy}
\label{sec:ble}

Bluetooth is a low-power, short-range wireless technology developed initially for replacing cables when
connecting devices like mobile phones, headsets, and computers.
It has since evolved into a wireless standard for connecting electronic devices to
form Personal Area Networks (PANs) and ad hoc networks. \cite{Dideles03}

Bluetooth is one of the most popular commodity radios for wireless devices.
As a representative of the frequency hopping spread spectrum radios,
it is a natural alternative to broadcast radios in the context of sensor networks. \cite{Leopold03}

The Bluetooth Special Interest Group (SIG) oversees
the Bluetooth standard and has a membership of over 38,000 companies as of 2023. \cite{bt_history}
The initial specification was made available in 1999.
Since then, the SIG released five core specifications, 5.4 being the latest adoption in 2023. \cite{bt_spec_history}

This thesis will focus mainly on Bluetooth Low Energy due to
its similarity in functionality and utilization to Zigbee and Thread/OpenThread.

blab bla.\cite{ble_primer23}.

\subsection{Introcution}
\label{ble:int}
The Bluetooth 4.0 Core Specification introduced Bluetooth Low Energy (BLE).
It is tempting to represent itself as a smaller and highly optimized version of
its bigger brother, classic Bluetooth. In reality, BLE has an entirely different lineage and design goals.
Initially known as Wibree and created by Nokia, the Bluetooth SIG ultimately embraced the technology.
The focus was to develop a radio standard with the lowest possible power consumption,
specifically optimized for low cost, low bandwidth, low power, and low complexity. \cite{Townsend14}

\subsection{Architecture}
\label{ble:ow}

According to Townsend et al., the Bluetooth Low Energy device can be categorized into
three primary blocks: controller, host, and application.
Each block has one or more layers, as shown in \autoref{fig:ble_arch}.

\begin{figure}[!ht]
    \centering
    \includegraphics[width=150mm, keepaspectratio]{figures/ble_arch_from_townsend.png}
    \caption{The BLE protocol stack. (Source: Figure 2-1 \cite{Townsend14})}
    \label{fig:ble_arch}
\end{figure}

The layers defined in \autoref{fig:ble_arch} provide the required functionality to operate.

The Controller contains the Physical Layer (PHY) and the Link Layer (LL).
These hold all the responsibility for operating the physical radio channel. 
Also, these layers are specialized by the LE standard keeping the Host
layer more-or-less the same as Bluetooth Classic.


A device address and an address type identify a device.
An address type indicates that an address can be either public or random.
Both are 48 bits long.
A device shall use at least one type of device address but can use
both simultaneously.
Public and random addresses are explained under the Link layer in section 2.4.

In BLE, Generic Access Profile (GAP) defines specific roles for each device.
These roles are: broadcaster, observer, peripheral, and central.
A device may support multiple roles but only one at a given time.
Each role specifies the requirements for the underlying Controller.
\autoref{ble:gap} explains GAP in more detail.


\subsection{Physical layer}
\label{ble:phy}

The physical layer (PHY) contains all necessary analog communication circuitry.
The radio uses the 2.4 GHz ISM band to communicate.
This band is divided into 40 channels or subbands from 2.400 GHz to 2.4835 GHz,
each subband using 2 MHz of bandwidth. (Yin et al., 2019)
This division can be seen in \autoref{fig:ble_phy}.
There are three dedicated advertising subbands for connection setup and
broadcast messages; the remaining 37 are known as general-purpose channels.

\begin{figure}[!ht]
    \centering
    \includegraphics[width=150mm, keepaspectratio]{figures/ble_phy.png}
    \caption{Channel allocation. (Source: Figure 2-2 \cite{Townsend14})}
    \label{fig:ble_phy}
\end{figure}

J. Yin et al. \cite{Yin:19} showed that from Core Specification 5, a novel
modulation scheme was introduced in the standard.
This feature allowed radio modules to use up all 2 MHz of bandwidth,
thus significantly increasing the maximum throughput.

\subsection{Link layer}
\label{ble:link}
Part B of the Core Specification's volume six (Low Energy Controller volume)
defines the Bluetooth Low Energy Link Layer's specification.

The Link Layer is the component that directly communicates with the PHY.
It is typically a blend of custom hardware and software solutions. \cite{Townsend14}

The hardware part usually implements all computationally intensive tasks,
such as CRC generation/validation, random number generation, and AES encryption. 

The software part of this layer is the only hard real-time part of
the whole stack because it has to match the radio hardware's timings
based on the PHY layer's specifications.

The Link Layer can be thought of as a state machine in operation.
In Figure 3, the representation of this state machine can be seen.
The Link Layer may have multiple instances of this state machine but shall have
at least one supporting Advertising or Scanning. (CoreSpec4_0, 2010)

The Core Specification 5.2 introduced two (2) new states:
Synchronization and Isochronous Broadcasting.

The Air Interface protocol defines the inner working of each state.
This interface consists of multiple access schemes, device discovery,
and Link Layer connection methods.

In the case of multiple state-machine instances, the first version of the
specification (4.0) prohibited certain combinations of states (and roles).
Table 1 specifies the allowed and prohibited state machine combinations.
From the release of Core Specification 5.2, an implementation may support
any combination of states and roles. The only criterion is that any two (2)
devices shall not have more than one connection between them. (CoreSpec5_2, 2019)

\subsubsection{Physical channel selection}
\label{ble:link:sel}
After the introduction in Core Specification 5, there are two algorithms for
channel selection.
The initial algorithm (called Channel Selection Algorithm #1) only supports
channel selection for connection events.
Algorithm #2 can be used for connection and periodic advertising events.
J. Yin et al. showed that this update significantly enhanced the robustness
of the protocol. (Yin, 2019)

\subsubsection{Link Layer Control protocol}
\label{ble:link:llcp}
The Link Layer Control protocol (LLCP) is used to control and
negotiate aspects of the operation of a connection between two Link Layers.
This includes procedures for control of the connection, starting and pausing
encryption, and other link procedures. (CoreSpec4_0, 2010)

One of the most significant procedures is the Channel Map Update Procedure.
This ensures that upon a physical channel update (initiated by
frequency hopping), both the master and the slave stay in sync
with the physical channel index.

% \usepackage{multirow}
\begin{table}[]
    \begin{tabular}{lllllll}
    \multicolumn{2}{l}{\multirow{2}{*}{Multiple State and Role}}  & \multirow{2}{*}{Advertising} & \multirow{2}{*}{Scanning} & \multirow{2}{*}{Initiating} & \multicolumn{2}{l}{Connection} \\
    \multicolumn{2}{l}{}                                          &                              &                           &                             & Master Role    & Slave Role    \\
    \multicolumn{2}{l}{Advertising}                               & Prohibited                   & Allowed                   & Allowed                     & Allowed        & Allowed       \\
    \multicolumn{2}{l}{Scanning}                                  & Allowed                      & Prohibited                & Allowed                     & Allowed        & Allowed       \\
    \multicolumn{2}{l}{Initiating}                                & Allowed                      & Allowed                   & Prohibited                  & Allowed        & Prohibited    \\
    \multicolumn{1}{c}{\multirow{2}{*}{Connection}} & Master Role & Allowed                      & Allowed                   & Allowed                     & Allowed        & Prohibited    \\
    \multicolumn{1}{c}{}                            & Slave Role  & Allowed                      & Allowed                   & Prohibited                  & Prohibited     & Prohibited   
    \end{tabular}
    \caption{Link Layer State Machine limitations (obsolate)}
    \label{tab:llstate_limit}
    \end{table}

\subsection{Host Controller Interface}
The Host Controller Interface (HCI) connects the Controller (low-level)
and Host (high-level) parts of the BLE stack.
It contains a technology-independent logical command interface structure
(HCI driver) and a Host Controller Transport Layer.
The HCI provides a uniform interface for accessing a Bluetooth Controller's
capability.

According to the standard, separating was necessary to keep the HCI driver
independent of the underlying transport technology.
Currently, in the latest standard (Core Specification 5.4), there are four (4)
different technologies defined for the transport layer:
UART, USB, Secure Digital (SD), and Three-wire UART.

\subsection{Logical Link Control and Adaption Protocol}
The Bluetooth logical link control and adaptation protocol (L2CAP)
supports higher-level protocol multiplexing, packet segmentation and
reassembly, and the conveying of quality of service information. 

Packet segmentation is necessary because data must fit into 27-byte maximum
payload size of the BLE packets on the transmit side. (Townsend, 14)
On the reception path, this procedure has to be reversed.

The Bluetooth logical link control and adaptation protocol (L2CAP) supports
higher-level protocol multiplexing, packet segmentation and reassembly,
and the conveying of quality of service information.
The protocol state machine, packet format, and composition are described
in this document. (Townsend, 14)


\subsection{Generic Access Profile}
\label{ble:gap}
The Generic Attribute Profile (GATT) builds on the Attribute Protocol (ATT)
and adds a hierarchy and data abstraction model on top of it.
In a way, it can be considered the backbone of BLE data transfer because
it defines how data is organized and exchanged between applications.



\subsection{Generic Attibute Profile}
\label{ble:gatt}

\section{IEEE 802.15.4}
\label{sec:15_4}

bla bla

\section{Zigbee}
\label{sec:zig}

Zigbee is a wireless communication protocol specifically designed for
applications requiring low-power consumption and short-range communication.
It operates on the IEEE 802.15.4 standard and provides a reliable and efficient way
for devices to connect and communicate with each other.
Zigbee is a frequently used technology in home automation, industrial control systems,
and various other applications related to the Internet of Things (IoT).
It supports mesh networking, allowing devices to relay data to extend the network
coverage. Zigbee is known for its low power consumption, simple implementation, and
ability to support a large number of devices in a network.

\url{https://www.academia.edu/73196763/A_Survey_of_ZigBee_Wireless_Sensor_Network_Technology_Topology_Applications_and_Challenges}

\url{https://issuu.com/ijtsrd.com/docs/399_literature_survey_on_zigbee_iee}
\url{https://www.sciencedirect.com/topics/engineering/zigbee-protocol}

\section{Thread/OpenThread}
\label{sec:ot}
openthread.

\url{https://ieeexplore.ieee.org/abstract/document/8767079}
\url{https://ieeexplore.ieee.org/abstract/document/8373620}
\url{https://ieeexplore.ieee.org/abstract/document/10022228}
\url{https://ieeexplore.ieee.org/abstract/document/9058170}
\url{https://dl.acm.org/doi/abs/10.1145/3507657.3528544}
\url{https://www.diva-portal.org/smash/record.jsf?pid=diva2%3A1040491&dswid=1593}
\url{https://upcommons.upc.edu/handle/2117/101928}
\url{https://ieeexplore.ieee.org/abstract/document/8106811}
\url{https://ieeexplore.ieee.org/abstract/document/8592759}
\url{https://ieeexplore.ieee.org/abstract/document/8374875}
\url{https://www.diva-portal.org/smash/record.jsf?pid=diva2%3A1282697&dswid=-3708}
\url{https://ieeexplore.ieee.org/abstract/document/8586865}
\url{https://link.springer.com/chapter/10.1007/978-3-030-70080-5_9#Sec8}