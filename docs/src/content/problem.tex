\chapter{Problem}
\label{chap:problem}

\section{Overview}
\label{sec:prob:ov}

Most vendors' flagship chips can handle two or more protocols simultaneously. For this problem definition, I want to create a solution for testing a triple protocol setup that can run BLE, Thread, and Zigbee side-by-side.

Testing should cover all network-level relations but does not have to validate full conformance. The tested application has the capability of a BLE peripheral device, a Minimal Thread Device, and a Zigbee End Device.

The foundation of this idea is that many IoT devices on the market can be configured or initially set via Bluetooth by the user and then work on a different network. Smart lightbulbs are great examples for this use case as they have a constant power supply. Both Bluetooth and Zigbee have defined a light control profile, but there are quite a few Thread-compatible smart lights and controllers.

The tested application would be able to act as a lightbulb on every network that it supports and does not require the user to dedicate him-/herself to invest in another network. The users can use whatever existing environment they have.


\section{Technologies}
\label{sec:prob:tech}

The solution should only depend on open-source solutions, as investing in tooling was not an option. Also, since Python is emerging as a future-proof and popular language, I chose it to be the language of the main codebase.

\section{Considerations}
\label{sec:prob:cons}

For the implementation of this study, Silicon Labs provided me with some wireless hardware\cite{silabs_wpk:2023}. I was given four of their EFR32xG24 Pro kits. It consists of a radio board - where the MCU is located - and a wireless Pro mainboard that can communicate with the radio board.

The mainboard can connect to the outside world via USB or a network connection via ethernet. A user can attach a remote shell via telnet if the board connects to a network. One can pass commands to the radio board on the remote prompt if the application has integrated CLI features.

Furthermore, Silicon Labs provided the necessary applications to run the scenario. The Silicon Labs' Gecko Software Development Kit (GSDK) is their integrated software solution. It is open-source and published on Git Hub\cite{silabs_gsdk:2023}. I was given a slightly modified version of the company's sample applications. All the SoC applications have a CLI interface compiled into the binary. Sadly, it is their internal solution, and I was given only the command specification.
