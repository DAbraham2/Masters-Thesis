%----------------------------------------------------------------------------
\chapter{\bevezetes}
%----------------------------------------------------------------------------

%A bevezető tartalmazza a diplomaterv-kiírás elemzését, történelmi előzményeit, a feladat indokoltságát (a motiváció leírását), az eddigi megoldásokat, és ennek tükrében a hallgató megoldásának összefoglalását.

%A bevezető szokás szerint a diplomaterv felépítésével záródik, azaz annak rövid leírásával, hogy melyik fejezet mivel foglalkozik.

\section{Problem definition}

%bla bla.
%
%IoT (\url{https://www.academia.edu/20619459/The_Internet_of_Things_IoT_An_Overview}, \url{https://www.academia.edu/43659334/Network_Protocols_Schemes_Mechanisms_for_Internet_of_Things_IoT_Features_and_Open_Challenges}, \url{https://www.academia.edu/49051619/EMERGING_WIRELESS_TECHNOLOGIES_IN_THE_INTERNET_OF_THINGS_A_COMPARATIVE_STUDY})
%Agriculture (\url{https://www.academia.edu/69562418/A_Survey_on_the_Role_of_IoT_in_Agriculture_for_the_Implementation_of_Smart_Farming}, \url{https://www.academia.edu/55117745/An_IoT_Based_Solution_for_Intelligent_Farming})


\section{Practical considerations}

\section{Thesis Outline}

This thesis has two parts: a literature review and a contribution. The literature review will consist of Chapters 2, 3, and 4.

In \autoref{chap:wireless}, Wireless protocols will be shown.
\autoref{chap:multiprot} will introduce the concept of multiprotocol networking, and \autoref{chap:testing} will introduce some software and system testing techniques.


\section{Contributions}
