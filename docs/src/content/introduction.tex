%----------------------------------------------------------------------------
\chapter{\bevezetes}
%----------------------------------------------------------------------------

%A bevezető tartalmazza a diplomaterv-kiírás elemzését, történelmi előzményeit, a feladat indokoltságát (a motiváció leírását), az eddigi megoldásokat, és ennek tükrében a hallgató megoldásának összefoglalását.

%A bevezető szokás szerint a diplomaterv felépítésével záródik, azaz annak rövid leírásával, hogy melyik fejezet mivel foglalkozik.

\section{Problem definition}
%IoT (\url{https://www.academia.edu/20619459/The_Internet_of_Things_IoT_An_Overview}, \url{https://www.academia.edu/43659334/Network_Protocols_Schemes_Mechanisms_for_Internet_of_Things_IoT_Features_and_Open_Challenges}, \url{https://www.academia.edu/49051619/EMERGING_WIRELESS_TECHNOLOGIES_IN_THE_INTERNET_OF_THINGS_A_COMPARATIVE_STUDY})

With the rapid spread of internet-of-things (IoT) devices worldwide, finding defects in protocol implementations has become more critical. Trying to define test cases or suites that truly move all the inner workings of a device is a delicate balance of complexity and resources. More automated solutions are necessary to help QA engineers and testers focus on solving problems and exotic cases.

In the last decade, the IoT market experienced a massive boom in revenue. According to Statista\cite{statista:revenue:2023}, from 2020 to 2023, revenue increased by 61 percent. The market is expected to grow to \$600 billion by the end of the current decade.

IoT is primarily used in large-scale sensor networks, making it suitable for Smart Cities, Utilities (energy management), and Smart Homes\cite{9210375}. Healthcare remote monitoring is expected to be another market where IoT could become a significant technology. The automotive industry and agriculture\cite{8883163, 9681084} (with the spread of precision agriculture) will also be great users.

According to Makhashari et al. \cite{9402092}, over 80 percent of developers face bugs at the device or the communication level. Out of these bugs, over 40 percent happen to be severe. They introduced taxonomy for bugs in IoT. For this thesis, the two main categories are device and connection bugs. Device bugs cover hardware and firmware issues. Connection bugs cover communication issues between multiple devices.

Previously, some studies discussed testing in the IoT realm. From functional testing\cite{10.1145/3528227.3528568, 10.1145/3611643.3613888} through pattern-based techniques\cite{10.1145/3278186.3278196} to AI guided techniques\cite{10.1145/3387940.3392218, 10.1145/3539637.3558049}. One of the studies discusses model-based techniques used in IoT testing\cite{10.1145/3425329.3425330}. There are two notable mentions of test ware setup in the literature \cite{10.1145/3479239.3485708, 10.1145/3368235.3368832}.


\section{Practical considerations}

With the rapid spread of IoT devices worldwide, finding defects in protocol implementations has become more critical. Trying to define test cases or suites that truly move all the inner workings of a device is a delicate balance of complexity and resources. More automated solutions are necessary to help QA engineers and testers focus on solving problems and exotic cases.

\section{Thesis Outline}

This thesis has two parts: a literature review and a contribution. The literature review will consist of Chapters 2, 3, and 4.

In \autoref{chap:wireless}, Wireless protocols will be shown.
\autoref{chap:multiprot} will introduce the concept of multiprotocol networking, and \autoref{chap:testing} will introduce some software and system testing techniques.


\section{Contributions}
During this thesis, I will create an abstract model to represent a valid IoT use case. Then, I present a solution to generate abstract test cases and eventually attach the test environment to the embedded system running in real time.
