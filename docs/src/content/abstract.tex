\pagenumbering{roman}
\setcounter{page}{1}

\selecthungarian

%----------------------------------------------------------------------------
% Abstract in Hungarian
%----------------------------------------------------------------------------
\chapter*{Kivonat}\addcontentsline{toc}{chapter}{Kivonat}

Az Internet of Things (IoT) eszközök gyors elterjedése új korszakot nyitott az összekapcsolt rendszerekben, forradalmasítva a környezetünkkel való interakcióinkat és a környezetünk észlelését egyaránt. Diplomamunkámban kiemelt hangsúlyt fektetek az IoT-protokoll rendszerek összetett feléptésének és működésének vizsgálatába - különös tekintettel a Bluetooth-ba, a Zigbee-be, valamint a Thread-be -, emellett foglalkozom a rendszerek átjárhatóságával a robusztus IoT-tesztelés módszertanát érintő kritikus kihívásokkal.

A kutatás középpontjában az IoT-eszközök meglévő tesztelési technikáinak értékelése áll, figyelembe véve a Bluetooth, a Zigbee és a Thread által támasztott egyedi követelményeket. Megvizsgálom a különböző tesztelési stratégiák hatékonyságát - beleértve a funkcionális, teljesítmény- és biztonsági tesztelést -,  annak érdekében, hogy feltárjam a különböző, az IoT-környezetek által támasztott kihívásokat.

A diplomamunkámban továbbá feltárom az IoT-tesztelés újonnan megjelenő módszereit, és javaslatot teszek egy egységes tesztelési keretrendszer létrehozására, amely alkalmazkodik a Bluetooth, a Zigbee és a Thread jellegzetes tulajdonságaihoz. Ezen protokollok egyetlen összefüggő tesztelési architektúrán belüli összehangolásával a kutatásom célja az IoT-ökoszisztémák átjárhatóságának javítása, elősegítve a megbízhatóbb és biztonságosabb összekapcsolt környezet létrehozását.


\vfill
\selectenglish


%----------------------------------------------------------------------------
% Abstract in English
%----------------------------------------------------------------------------
\chapter*{Abstract}\addcontentsline{toc}{chapter}{Abstract}

The rapid proliferation of Internet of Things (IoT) devices has ushered in a new era of interconnected systems, revolutionizing the way we interact with and perceive our surroundings. This master's thesis delves into the intricacies of IoT protocols, with a specific focus on Bluetooth, Zigbee, and Thread, addressing the critical challenges associated with their interoperability and the methodologies employed for robust IoT testing.

The core of the research revolves around the evaluation of existing testing techniques for IoT devices, considering the unique requirements posed by Bluetooth, Zigbee, and Thread. Comprehensive testing strategies, including functional, performance, and security testing, are scrutinized to uncover their efficacy in addressing the challenges posed by diverse IoT environments.

Furthermore, the thesis explores the emerging methodologies for IoT testing and proposes a unified testing framework that accommodates the distinctive features of Bluetooth, Zigbee, and Thread. By aligning these protocols within a cohesive testing architecture, the research aims to enhance the overall interoperability of IoT ecosystems, fostering a more reliable and secure interconnected landscape.



\vfill
\cleardoublepage

\selectthesislanguage

\newcounter{romanPage}
\setcounter{romanPage}{\value{page}}
\stepcounter{romanPage}