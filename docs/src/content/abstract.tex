\pagenumbering{roman}
\setcounter{page}{1}

\selecthungarian

%----------------------------------------------------------------------------
% Abstract in Hungarian
%----------------------------------------------------------------------------
\chapter*{Kivonat}\addcontentsline{toc}{chapter}{Kivonat}

Az Internet of Things (IoT) eszközök gyors elterjedése az összekapcsolt rendszerek új korszakát nyitotta meg, forradalmasítva a környezetünkkel való interakciót és a környezet észlelését. Ez a diplomamunka bevezet az IoT-protokollok bonyolultságába, különös tekintettel a Bluetooth-ba, a Zigbee-be és a Thread-be, és foglalkozik az interoperabilitással és a robusztus IoT-tesztelés módszertanával kapcsolatos kritikus kihívásokkal.

A kutatás lényege az IoT-eszközök meglévő tesztelési technikáinak értékelése körül forog, figyelembe véve a Bluetooth, a Zigbee és a Thread által támasztott egyedi követelményeket. Átfogó tesztelési stratégiákat, beleértve a funkcionális, a teljesítmény- és a biztonsági tesztelést, alaposan megvizsgálják, hogy feltárják hatékonyságukat a különféle IoT-környezetek által támasztott kihívások kezelésében.

Ezenkívül a dolgozat feltárja az IoT tesztelésének feltörekvő módszereit, és javaslatot tesz egy olyan egységes tesztelési keretrendszerre, amely alkalmazkodik a Bluetooth, a Zigbee és a Thread jellegzetes tulajdonságaihoz. E protokollok egy összefüggő tesztelési architektúrán belüli összehangolásával a kutatás célja az IoT-ökoszisztémák átfogó interoperabilitásának javítása, elősegítve a megbízhatóbb és biztonságosabb összekapcsolt környezetet.



\vfill
\selectenglish


%----------------------------------------------------------------------------
% Abstract in English
%----------------------------------------------------------------------------
\chapter*{Abstract}\addcontentsline{toc}{chapter}{Abstract}

The rapid proliferation of Internet of Things (IoT) devices has ushered in a new era of interconnected systems, revolutionizing the way we interact with and perceive our surroundings. This master's thesis delves into the intricacies of IoT protocols, with a specific focus on Bluetooth, Zigbee, and Thread, addressing the critical challenges associated with their interoperability and the methodologies employed for robust IoT testing.

The core of the research revolves around the evaluation of existing testing techniques for IoT devices, considering the unique requirements posed by Bluetooth, Zigbee, and Thread. Comprehensive testing strategies, including functional, performance, and security testing, are scrutinized to uncover their efficacy in addressing the challenges posed by diverse IoT environments.

Furthermore, the thesis explores the emerging methodologies for IoT testing and proposes a unified testing framework that accommodates the distinctive features of Bluetooth, Zigbee, and Thread. By aligning these protocols within a cohesive testing architecture, the research aims to enhance the overall interoperability of IoT ecosystems, fostering a more reliable and secure interconnected landscape.



\vfill
\cleardoublepage

\selectthesislanguage

\newcounter{romanPage}
\setcounter{romanPage}{\value{page}}
\stepcounter{romanPage}