\chapter{Conclusions}
\label{chap:conclusion}
\section{Summary}
\label{sec:conc:summ}
In this thesis, I made a literature review on currently used IoT network protocols. After the protocols, I conducted a search for testing strategies applicable to use in IoT testing.

After finishing the literature review, I presented a small but valid use case. Then, I constructed an abstract model representing the scenario. I have managed to create abstract test cases automatically from the abstract model. I developed a Python library to map abstract test cases to concrete tests. This library controls and monitors the System Under Test and the surrounding helper nodes.

Lastly, I managed to containerize the running environment to be platform-independent. Next, I tested the provided application and summarized the test results.

\section{Results}
\label{sec:conc:disc}
After conducting a thorough literature review on different protocols and testing methodologies, I concluded that using model-based techniques could be beneficial in testing IoT protocol implementations. It is a valuable technology that, when used correctly, can reduce test design time and costs. Model-based testing can automatically create test suites to cover different aspects of the system.

During the thesis, I managed to find new defects in an already existing and tested product. This shows yet again that model-based testing could discover new edge cases without a tester's manual intervention or supervision.

\section{Future work}
\label{sec:conc:future}
For future work, I would model more aspects of the application. Expand the model with the application-level logic, especially using BLE GATT or Zigbee Cluster services to test more profound and broader functionalities.

As graphwalker has limited capabilities in modeling, I recommend creating a translator that can transform more sophisticated models (i.e., UML state machine) into a directed graph model parsed by graphwalker. This way, the tester can use higher abstractions to model the problem with composite states.